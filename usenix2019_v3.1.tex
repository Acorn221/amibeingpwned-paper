%%%%%%%%%%%%%%%%%%%%%%%%%%%%%%%%%%%%%%%%%%%%%%%%%%%%%%%%%%%%%%%%%%%%%%%%%%%%%%%%
% Template for USENIX papers.
%
% History:
%
% - TEMPLATE for Usenix papers, specifically to meet requirements of
%   USENIX '05. originally a template for producing IEEE-format
%   articles using LaTeX. written by Matthew Ward, CS Department,
%   Worcester Polytechnic Institute. adapted by David Beazley for his
%   excellent SWIG paper in Proceedings, Tcl 96. turned into a
%   smartass generic template by De Clarke, with thanks to both the
%   above pioneers. Use at your own risk. Complaints to /dev/null.
%   Make it two column with no page numbering, default is 10 point.
%
% - Munged by Fred Douglis <douglis@research.att.com> 10/97 to
%   separate the .sty file from the LaTeX source template, so that
%   people can more easily include the .sty file into an existing
%   document. Also changed to more closely follow the style guidelines
%   as represented by the Word sample file.
%
% - Note that since 2010, USENIX does not require endnotes. If you
%   want foot of page notes, don't include the endnotes package in the
%   usepackage command, below.
% - This version uses the latex2e styles, not the very ancient 2.09
%   stuff.
%
% - Updated July 2018: Text block size changed from 6.5" to 7"
%
% - Updated Dec 2018 for ATC'19:
%
%   * Revised text to pass HotCRP's auto-formatting check, with
%     hotcrp.settings.submission_form.body_font_size=10pt, and
%     hotcrp.settings.submission_form.line_height=12pt
%
%   * Switched from \endnote-s to \footnote-s to match Usenix's policy.
%
%   * \section* => \begin{abstract} ... \end{abstract}
%
%   * Make template self-contained in terms of bibtex entires, to allow
%     this file to be compiled. (And changing refs style to 'plain'.)
%
%   * Make template self-contained in terms of figures, to
%     allow this file to be compiled. 
%
%   * Added packages for hyperref, embedding fonts, and improving
%     appearance.
%   
%   * Removed outdated text.
%
%%%%%%%%%%%%%%%%%%%%%%%%%%%%%%%%%%%%%%%%%%%%%%%%%%%%%%%%%%%%%%%%%%%%%%%%%%%%%%%%

\documentclass[letterpaper,twocolumn,10pt]{article}
\usepackage{usenix2019_v3}

% to be able to draw some self-contained figs
\usepackage{tikz}
\usepackage{amsmath}

% inlined bib file
\usepackage{filecontents}

%-------------------------------------------------------------------------------
\begin{filecontents}{\jobname.bib}
%-------------------------------------------------------------------------------
% TODO: Add references as needed
\end{filecontents}

%-------------------------------------------------------------------------------
\begin{document}
%-------------------------------------------------------------------------------

%don't want date printed
\date{}

% make title bold and 14 pt font (Latex default is non-bold, 16 pt)
\title{\Large \bf Am I Being Pwned? A Hybrid Static-Analysis and LLM Pipeline\\
  for Browser Extension Security}

%for single author (just remove % characters)
\author{
{\rm Your N.\ Here}\\
Your Institution
\and
{\rm Second Name}\\
Second Institution
% copy the following lines to add more authors
% \and
% {\rm Name}\\
%Name Institution
} % end author

\maketitle

%-------------------------------------------------------------------------------
\begin{abstract}
%-------------------------------------------------------------------------------

% TODO: ~150-200 words
% - Browser extensions have deep access, CWS review is insufficient
% - We present a hybrid pipeline: AST data-flow analysis + LLM triage
% - Key results: X extensions analyzed, Y flagged, Z confirmed malicious

\end{abstract}


%-------------------------------------------------------------------------------
\section{Introduction}
\label{sec:intro}
%-------------------------------------------------------------------------------

% TODO:
% - Browser extensions have deep access, users trust blindly
% - CWS review process is insufficient at scale
% - Contribution: hybrid pipeline combining AST data-flow analysis + LLM triage
% - Key findings teaser (X extensions analyzed, Y% flagged, Z confirmed)
% - Paper outline


%-------------------------------------------------------------------------------
\section{Background and Threat Model}
\label{sec:background}
%-------------------------------------------------------------------------------

% TODO:
% - Extension architecture (content scripts, background/service workers,
%   permissions model, message passing)
% - Known attack patterns (data exfil, credential theft, proxy abuse,
%   extension enumeration)
% - Attacker model: what a malicious extension author can do
% - Why existing CWS review falls short


%-------------------------------------------------------------------------------
\section{System Design}
\label{sec:design}
%-------------------------------------------------------------------------------

% TODO: Pipeline overview diagram
% (download -> deobfuscate -> static analysis -> LLM triage -> DB)

%-----------------------------------
\subsection{Data Collection}
\label{sec:collection}
%-----------------------------------

% TODO:
% - CWS scraping methodology, 25K+ extensions
% - Category and search-based discovery
% - CRX download and extraction

%-----------------------------------
\subsection{Deobfuscation}
\label{sec:deobfuscation}
%-----------------------------------

% TODO:
% - jsbeautifier preprocessing
% - Why this step matters for downstream analysis

%-----------------------------------
\subsection{Static Analysis}
\label{sec:static}
%-----------------------------------

% TODO: Core technical contribution
% - Babel AST parsing, source/sink identification
% - Intra-file taint propagation
% - Constant propagation (defeating window[x]() obfuscation)
% - Cross-component flow stitching via message passing (comm-graph)
% - Risk scoring model (exfil flows, code exec, WASM, permissions)

%-----------------------------------
\subsection{LLM-Assisted Triage}
\label{sec:llm}
%-----------------------------------

% TODO:
% - How LLM agents consume static analysis output
% - Structured report format (risk, summary, flagCategories, endpoints)
% - What the LLM adds vs what static analysis alone catches
% - Cost and throughput considerations


%-------------------------------------------------------------------------------
\section{Implementation}
\label{sec:impl}
%-------------------------------------------------------------------------------

% TODO:
% - Tech stack (TypeScript/Babel analyzer, Python pipeline, PostgreSQL)
% - Scale numbers, performance characteristics
% - Database schema, web-accessible resource fingerprinting
% - Practical considerations (RAM on large extensions, batch processing)


%-------------------------------------------------------------------------------
\section{Evaluation}
\label{sec:eval}
%-------------------------------------------------------------------------------

%-----------------------------------
\subsection{Dataset}
\label{sec:dataset}
%-----------------------------------

% TODO:
% - Extension corpus stats (categories, user counts, permissions distribution)

%-----------------------------------
\subsection{Static Analyzer Accuracy}
\label{sec:eval-static}
%-----------------------------------

% TODO:
% - Benchmark results
% - Exfil flows as best discriminator (74% vs 26%)
% - False positive/negative rates
% - Comparison of flag categories by signal strength

%-----------------------------------
\subsection{LLM Triage Quality}
\label{sec:eval-llm}
%-----------------------------------

% TODO:
% - Agreement with manual review
% - What LLM catches that static misses (and vice versa)

%-----------------------------------
\subsection{End-to-End Pipeline}
\label{sec:eval-e2e}
%-----------------------------------

% TODO:
% - Throughput, cost analysis
% - Comparison to manual-only analysis

%-----------------------------------
\subsection{Comparison with Prior Work}
\label{sec:eval-comparison}
%-----------------------------------

% TODO:
% - CRXcavator, Extension Monitor, etc.


%-------------------------------------------------------------------------------
\section{Findings}
\label{sec:findings}
%-------------------------------------------------------------------------------

% TODO:
% - Prevalence of malicious/vulnerable extensions in CWS
% - Most common attack patterns (data exfil, hardcoded secrets, postMessage)
% - Case studies of the most interesting/dangerous extensions
% - Responsible disclosure outcomes


%-------------------------------------------------------------------------------
\section{Discussion}
\label{sec:discussion}
%-------------------------------------------------------------------------------

% TODO:
% - Limitations (what the pipeline misses, evasion techniques)
% - LLM cost/accuracy tradeoffs
% - Recommendations for CWS/Google
% - Generalizability to other extension ecosystems (Firefox, Edge)


%-------------------------------------------------------------------------------
\section{Related Work}
\label{sec:related}
%-------------------------------------------------------------------------------

% TODO:
% - Prior browser extension security studies
% - Static analysis tools for JS/extensions
% - LLM-assisted code analysis


%-------------------------------------------------------------------------------
\section{Conclusion}
\label{sec:conclusion}
%-------------------------------------------------------------------------------

% TODO:
% - Summarize contributions and key findings


%-------------------------------------------------------------------------------
\section*{Acknowledgments}
%-------------------------------------------------------------------------------

% TODO

%-------------------------------------------------------------------------------
\section*{Availability}
%-------------------------------------------------------------------------------

% TODO:
% - Open-source tooling, dataset availability

%-------------------------------------------------------------------------------
\bibliographystyle{plain}
\bibliography{\jobname}

%%%%%%%%%%%%%%%%%%%%%%%%%%%%%%%%%%%%%%%%%%%%%%%%%%%%%%%%%%%%%%%%%%%%%%%%%%%%%%%%
\end{document}
%%%%%%%%%%%%%%%%%%%%%%%%%%%%%%%%%%%%%%%%%%%%%%%%%%%%%%%%%%%%%%%%%%%%%%%%%%%%%%%%

%%  LocalWords:  endnotes includegraphics fread ptr nobj noindent
%%  LocalWords:  pdflatex acks
